\chapter{Research Questions}\label{chapter:ResearchQuestions}

\section{Initial Hypothesis}

\indent A software architecture document is not just a necessary afterthought of architecture design (\cite{BachmannDocumentingSoftware2010}), but an important contributor to the entire software design and development lifecycle.  At an initial phase, for a new project, the software architecture document is produced as an artifact for software architecture views for different stakeholders. During the course of project lifecycle, the software architecture document grows and serves as an artifact to record important architectural decision made by the architects. At the design phase the software architecture document provides developers with a high level view of the software architecture and helps to understand the system interfaces, component interaction and basic functionality of each architectural component. 
\\\indent A software architecture document is not a static artifact. Rather, it is as dynamic as the software requirements itself (\cite{BachmannDocumentingSoftware2010}). Maintenance of software architecture requires deep understanding of the skeleton system and depends heavily on its documentation. This escalates documentation to the highest position in the software evolution cycle. But usefulness of this document is measured by its relevance and consistency. This requires maintenance of the document itself to keep it as up-to-date as the current system. Thus, software architecture documentation is an integral activity that revolves not only at a software inception phase, during software architecture design, but also during the course of software’s development maintenance and evolution. Since documentation is an activity, it needs to be regulated as a software process. 
\\\indent Software process is affected by organizational behavior of a community (\cite{Fuggeffa1988}). Different organizations work on a culture specific to the standards and processes followed by the within their scope of control. In this context, Open Source software communities are noteworthy due to their relaxed process control and organizational structure. With regards to any form of artifact, especially documentation, this community is loosely coordinated where developers or contributors tend to code solutions without producing adequate documentation (\cite{6923128}).
\\\indent This brings us to an initial hypothesis that forms the basis for this research work on Software architecture documentation process: Open source software community lacks a process for maintenance of software architecture documentation. 
For a concrete example, Mediawiki was chosen as the ideal candidate. In the last few years the wiki community (WMF - Wikimedia Foundaion) become one of the largest open source communities in the world. The software that runs the these wikis is the Mediawiki engine. The robust architecture of the mediawiki software is a complex system that has evolved over the years and its architecture complexity has grown manifolds. To explain its architecture, some documentation is available on “mediawiki.org”. But to cater to new developers and first time users of mediawiki, architecture details and technicalities of architectural components is scarcely available on “mediawiki.org”. Although some architectural component documentation is available as a part of the source code, this documentation is not well structured or available in wiki format. This deficit was realized as a part of the initial study and discussions with the stakeholders at mediawiki which will be elaborated in \autoref{CurrentSOA}. Hence, all the research and conceptualization of improved documentation process is based on these initial ideas. The following section lists the research questions that are intended to be answered by this thesis work.

\section{Research Questions}
\begin{enumerate}
\item RQ1\label{RQ1}  : How software architecture documentation process can be improved for Wikimedia Software?
\item RQ2\label{RQ2} : What state-of-the-art architecture documentation process (methodology, tools) are available in the industry that meet domain-specific requirements – e.g. Open Source Software ?
\item RQ3\label{RQ3} : What are the quality characteristics and metrics for evaluation of the software architecture documentation process?
\item RQ 4\label{RQ4} : Which specific requirements of Wikimedia stakeholders should be met by documentation process for Mediawiki Software Architecture Documentation ?
\item RQ 5\label{RQ5} : What process can be followed to automate the quality assurance of Software Architecture documentation in Open Source Software
\end{enumerate}

The following sections will explain the reasons and requirements that lead to the formulation of the above-mentioned research questions : 
 
\section{Current state-of-art}\label{CurrentSOA} 
\indent The following sub-sections explain the current state of mediawiki software architecture documents and the current software and documentation process in the organization.
\subsection{Software Architecture Documents}
\indent Mediawiki currently has all its software architecture documentation available on \enquote{mediawiki.org}. The wiki pages belong to different namespaces such as \enquote{Manual:}, \enquote{Help:} etc. to segregate them according to the intended information. Yet, these documents are scattered as a forest of links like any typical wiki, which makes it hard to follow for new users and harder to maintain for the existing users. 
\\\indent The available documents are useful for understanding some architecture components and help new mediawiki users to understand their installation, usage and operational details. But, these documents are not detailed enough for new developers to acquire a thorough understanding of the architectural component.  Documentation of the Mediawiki core source code is auto-generated via \enquote{Doxygen} and is available at \emph{https://doc.wikimedia.org/mediawiki-core/master/php }. This auto-generated code level documentation is always updated as a part of cron-jobs during deployment cycles and hence they are auto-maintained. The overview of Mediawiki's architecture was captured and written as a part of the book \enquote{The Architecture of Open Source Applications} by \emph{Sumana Harihareshwara} and \emph{Guillaume Paumier}. This documentation, availbale on \enquote{Mediawiki.org}, is an excellent explanation of the various architectural decisions and corresponding rationale that were adopted over the years leading to the current architectural state  of Mediawiki software. It is available under the \enquote{Manual:} namespace on \enquote{Mediawiki.org} and can be viewed for an abstract high level understanding of the system.
\begin{figure}[H]
  \centering
  \includegraphics[width=100mm]{images/arch_doc_level}
  \caption[Documentation available for software architecture levels]{Current state of documentation for different software architecture levels.}\label{fig:ArchDocLevel}
\end{figure}
\indent In \autoref{fig:ArchDocLevel}  we can see the current documentation structure that is available for different levels of detail of the software architecture. The green area indicated the existing documentation and the red area in the indicates the lack of availability and maintenance of complete architecture documentation in accordance with the standard software architecture documentation structure \cite{BachmannDocumentingSoftware2010} which covers the different views, rationale,  etc.
\subsection{Software Process}
\indent Mediawiki software community follows a process for maintaining its software (code base) that involves the interaction of the multiple systems for its review, versioning, tracking and task management. In this regard, before a piece of code is deployed into production environment, it is important to understand the role of the following entities as a part of the software process. 
\begin{enumerate}
\item Developers : The software developers are the the most important functional entity of the software process in any software project or organisation. Similarly in the mediawiki community, the process is driven, managed and used by the developers of the software. Although other roles like software architect may exist as a subset of the stakeholders within the community, they all belong to the larger set of \enquote{Developers}. Developers have the ultimate responsibility to implement and manitain the software process.
\item Maintainers : As the name suggests, Maintainers are developers with the acquired competence to take up the responsibility and become maintainers of different modules in the mediawiki code base \emph{https://www.mediawiki.org/wiki/Developers/Maintainers}. They are instrumental in reviewing and following the software process and help to track and complete required functionality
\item Mediawiki BOTs : Besides human maintainers, Bots assume the role of semi-automated process to carry out maintenance activities that may be time-consuming or impossible to perform manually \emph{https://www.mediawiki.org/wiki/Project:Bots}.
\end{enumerate}
\indent The above-mentioned entities need supporting systems to perform their daily activities as a part of the software process. In mediawiki, the software process activities are supported by following systems that simplify the process management  activities. 
\begin{enumerate}
\item Gerrit : Gerrit (\emph{https://code.google.com/p/gerrit }) is a web-based code collaboration tool that has been adopted by the mediawiki community for managing the code base. This tool allows the review and maintenance of the master and forked branches of the mediawiki code repository and allows the developers to manage their contributions. The tools allows code management as a part of the software process of mediawiki which helps in easy maintenance of the software. 
\item Phabricator : Phabricator (\emph{http://phabricator.org/}) is an open-source task management and project communication platform that helps to manage different projects and their stakeholders within the organization. The mediawiki community has adopted the Phabricator to manage their daily tasks related to software development. The tasks can be managed according to projects, build versions, tags, etc by human maintainers. It provides features to discuss on issues related to the task and to also fork new related tasks.
\end{enumerate}
\indent \autoref{fig:MWSoftwareProcess}  shows the sequence diagram that explains a simple use case scenario : A software development task and the process followed for task management. 
\\\indent A developer may create a task on Phabricator to add/update a software functionality. He assigns the task either to himself or to another developer. The developed piece of code is pushed to an intermediate repository in Gerrit and awaits review. Once the code is reviewed and approves by senior developers, it is pushed to the authoritative repository which is ready for deployment. Once this is completed the task is finally closed.  
\begin{figure}[H]
  \centering
  \includegraphics[width=100mm]{images/MWSoftwareProcess}
  \caption[Current software maintenace process Sequence diagram]{Mediawiki Software Process Sequence diagram.}\label{fig:MWSoftwareProcess}
\end{figure}
\subsection{Documentation Process}
\indent Similar to their software process, the mediawiki community has a standard software architecture documentation process which involves the interaction of human maintainers and use of Phabricator for task management. Tasks for documentation activity are created manually, based on the need realized by developers. The management of the task is manual and its tracking, organization and management is supported by Phabricator
\\\indent \autoref{fig:MWDocProc} sequence diagram explains the use-case scenario : Manage a documentation task to update a document on \enquote{mediawiki.org}
\\\indent In this case a developer himself may create/ assign a task on Phabricator for document update on \enquote{mediawiki.org}. Once the update has been completed, a the task maintainer comments and closes the task on Phabricator.
\begin{figure}[H]
  \centering
  \includegraphics[width=100mm]{images/MWDocProc}
  \caption[Current documentation process Sequence diagram]{Mediawiki Documentation Process Sequence diagram.}\label{fig:MWDocProc}
\end{figure}
\indent Understanding the current software documentation process leads to the following inherent problems and required improvements that need to be catered by answering the research questions

\section{Problems}
\indent This section elaborates on the problems that have been identified in the software architecture documentation process of Mediawiki that call for an improvement in the documentation process (RQ1).
\subsection{Maintainability}
\indent As seen in the scenario covered in the previous section, it is evident that the documentation has shortcomings in terms of its maintainability with the rapid evolution of the software architecture itself. The process followed by the community is not strictly structured to ensure that the documents are maintained up-to-date. Phabricator may help to organize the task of documentation but does not guarantee the availability of precise documentation itself. Also only a manual check on document maintenance, without a strict process, is highly dependent on the motivation of the task owner to create, assign and complete the task. With the existing documentation process, a key requirement of document maintainability is not completely satisfied. Hence there is a need for an improvement to incorporate the requirement of up-to-date documents as a part of the documentation process.
\subsection{Roles and Responsibilities}
\indent Mediawiki is an open source software community and hence it is not structured in its organization of well-defined roles and responsibilities. This poses a problem in defining, maintaining and following a strict process-based approach for software development and documentation. As compared to code maintainers mentioned in the previous section, there is no defined responsibility in the mediawiki community specially focused on documentation. The role of a developer for a certain architectural component implicitly assigns him the responsibility of corresponding documentation maintenance. But the lack of explicitly defined responsibility for the same creates a relaxed documentation process.
\subsection{Availability and Management}
\indent An issue with the current documentation process is that software architecture documentation is not available under a single namespace or category and rather scattered in the wiki-forest. This makes it harder to manage the documentation and guarantee its availability on \enquote{mediawiki.org}.
\section{Requirement Analysis}
\indent The above listed problems were identified to understand the requirements to be met by the improved process (RQ1).
\subsection{Stakeholders}
\indent To understand a system and its architecture, it is important to understand the stakeholder perspective (RQ4).  Mediawiki's software architecture documentation is available for developers, architects and system administrators on \enquote{mediawiki.org}. Out of this the developers are the largest stakeholder group that access and use the architecture documents to the maximum. To cater to new developers various channels and features offer help in the form of mailing lists, IRC (Internet Relay Channel), Feedback dashboard, etc. 
\\\indent But a more concrete documentation needs to be prepared and maintained by the architecture component developers themselves. These detailed documents will help future developers to understand the software architecture in a more comprehensive way and on a more readable medium (mediawiki.org). This requirement was also realized during the \enquote{Mediawiki Developer meetup – 2009} which suggested the need for improved documentation and hinted on the usage of Bots for maintenance purpose.
\\\indent Stakeholders play an important role in the implementation and maintenance of a process. Likewise in the case of documentation process, the developers are the key stakeholders who as both provider and user of the documents. As the developers understand their respective development in the best possible way, they themselves should prepare the documentation for the corresponding component/ feature/ module. This will help to capture the architecture decisions and rationale that can be utilized for future reference.
\subsection{Meetings / Interactive Sessions}
\indent To understand the requirements from the perspective of stakeholders, sessions were held remotely and on-site with the members of the mediawiki organization at Berlin. These meetings and conversations gave a chance to understand the existing process and requirements for process improvement in a more detailed and focused manner (RQ2).
The mediawiki representatives explained that although a compressed user guide could be copied along with a fresh wiki installation that includes basic information/ details in a concise yet understandable form, there is dire lack of a structured, detailed and complete architecture documentation within the community (RQ4). 
\\\indent Some documentation is available as a part of the source code for some architectural components. But the community prefers to have all documentation available on \enquote{mediawiki.org}(RQ4).
\\\indent The problem that documentation is often not updated / maintained due to lack of a strict process was realized within the community who wanted quality documents that were mostly up-to-date (RQ3). A process that streamlines this maintenance activity was put up as an important requirement during these meetings (RQ4). 
\\\indent The availability of guidelines to support the preparation of software architecture documents and assigning responsibility of its maintenance to developers or bots will assure the quality of the resulting documents (RQ5). The documentation is best understood and evaluated by the developers using them and thus, quality of  documentation was indicated as an important requirement.
