\chapter{Conceptualization}\label{chapter:Conceptualization}
\indent After framing of research question and sorting ideas from related work and literature, the thesis contribution aims to answer the questions posed initially and conceptualize an optimal solution. The solution should not only meet the stakeholder and community requirements but also confirm to some already established / deployed standards or tools. The ultimate aim is to find a solution that is evaluated and accepted by Mediawiki stakeholders as a deployable/ usable solution. 
\section{Idea Generation and Evolution}
\indent This section outlines the approach that was taken to understand the existing system and derive solutions for improved documentation process. 

\subsection{Preparatory Tasks}
\indent During the initial weeks of conceptualization phase discussion were held and emails were exchanged with the developers / architects at Mediawiki to come to an understanding of their current community setup and to get a beginner's guide to the system. Their suggestions and the advise of experienced developers to kick-start included the following :
\begin{itemize}
\item Analyze the documentation that already exists on \enquote{Mediawiki.org} to get a high level understanding of the system and some low-level implementation details of certain components which are well-documented (e.g) Mediawiki APIs (\emph{https://www.mediawiki.org/wiki/API:Tutorial}), Extensions (\emph{https://www.mediawiki.org/wiki/Manual:Extensions}), etc.
\item Look into \enquote{Doxygen} tool for documenting software architecture to check the level of details captured by auto-generated documentation.
\item When understanding a software system, it is important to understand the \enquote{Who, What and How ?}. This refers to the identification and understanding of current roles, responsibilities and processes that are practiced in the community.
\item Process for generating and maintaining documentation by mediawiki community was important to understand in order to know its shortfalls.
\item It was suggested that in order to understand the wiki platform \enquote{mediawiki.org} it was important to use various features like templates, extensions, visualizations.
\end{itemize}



\subsection{Initial ideas }

Different versions - arguments and decision-making, user scenarios discussions


Using wiki for documentation disadvantage:
Review not possible
offline work not possible
Tracking changes not possible
A page save creates a new history entry. This may lead to unnecessary revision history 

\section{Improved Process}
Documentation health monitor
Review of documents in a process-oriented structure - Gerrit review 

\subsection{Roles and Responsibility definition and co-ordination}

Key principles to address challenges of the task-centered collaboration approach are [Felix Master thesis]
1.) the self-organization of the community through task decomposition, 
2.) an on-line community support based on social design principles and best practices and
 3.) an open science process to enable unanticipated contributions

	Solves issues identified previously
 	
 	streamline more people into a process
 	
 figure	
 
 \begin{figure}[H]
  \centering
  \includegraphics[width=100mm]{images/Process_Roles}
  \caption[Defining distinct roles and responsibilities in a process]{Defining distinct roles and responsibilities in a process.}\label{fig:Process_Roles}
\end{figure}

 \begin{figure}[H]
  \centering
  \includegraphics[width=100mm]{images/role_Maintainers}
  \caption[The sphere of Maintainer's roles and responsibilities]{The sphere of Maintainer's roles and responsibilities.}\label{fig:role_Maintainers}
\end{figure}

\subsection{Document Maintenance Bot}
Using bot to take on the responsibility of 
maintainable/ visible


\subsection{A proof of concept}
figure : Bot in the system

In \autoref{fig:MWDocBot}  we can see the 
\begin{figure}[H]
  \centering
  \includegraphics[width=\textwidth]{images/MWDocBot}
  \caption[Maintenace Bot Sequence diagram]{Maintenace Bot Sequence diagram.}\label{fig:MWDocBot}
\end{figure}