\chapter{Conceptualization}\label{chapter:Conceptualization}
\indent After framing of research question and sorting ideas from related work and literature, the thesis contribution aims to answer the questions posed initially and conceptualize an optimal solution. The solution should not only meet the stakeholder and community requirements but also confirm to some already established / deployed standards or tools. The ultimate aim is to find a solution that is evaluated and accepted by Mediawiki stakeholders as a deployable/ usable solution. 
\section{Idea Generation and Evolution}
\indent This section outlines the approach that was taken to understand the existing system and derive solutions for improved documentation process. 

\subsection{Preparatory Tasks}
\indent During the initial weeks of conceptualization phase discussion were held and emails were exchanged with the developers / architects at Mediawiki to come to an understanding of their current community setup and to get a beginner's guide to the system. Their suggestions and the advise of experienced developers to kick-start included the following :
\begin{enumerate}
\item \textbf{Understand and use the Software : } It was important to download and locally install Mediawiki open source software to understand the basic components of the software architecture. The installation guide \cite{Installation_guide} helped in understanding the system requirements and successfully configuring and setting up the executable code on the \enquote{localhost} under the the name \enquote{en} to denote the English language version of the setup. 
\begin{figure}[H]
  \centering
  \includegraphics[width=100mm]{images/code_stats}
  \caption[Mediawiki code statistics \cite{openhub_MW}]{Mediawiki code statistics \cite{openhub_MW}.}\label{fig:code_stats}
\end{figure}
The Mediawiki open source package is written in PHP \cite{mediawiki} (originally written for use on Wikipedia) - \autoref{fig:code_stats} \cite{openhub_MW}. The first approach to understand the architecture was to dive into the \enquote{\textbackslash{includes}} folder which comprises the code for the basic architectural components of the Mediawiki software like \enquote{api}, \enquote{cache}, \enquote{db} and many more. 
\indent On successful setup of Mediawiki software and configuring the necessary database and server setup, the main page of the local installation powered by the Mediawiki engine can be launched as seen in the \autoref{fig:Mediawiki_mainpage}.
This complete setup now helped to play around the software and the wiki to understand, use and modify its features.
\item \textbf{Analyze the documentation that already exists : } It was important to understand the documentation that is already available on \enquote{Mediawiki.org} in order to analyze the pros and cons of the existing structure and process of the software architecture documentation. The existing documentation helped to get a high level understanding of the system and some low-level implementation details of certain components which are well-documented (e.g) Mediawiki APIs \cite{api_tutorial}, Extensions \cite{manual_extensions}, etc. Documentation is also available for developers and system administrators \cite{MW_doc}
\\\indent  During code analysis the important architectural components and modules were identified as candidates that were in need of an improved, organized, detailed and structured architecture documentation :
\begin{itemize}
\item Installation, Update and Deployment
\item Page processing, parsing, rendering and caching
\item Extensions
\item Architecture and Software Performance
\item Internationalization / Localization
\item Static and Dynamic Structure
\item User and Access Control
\item Database Design
\item Security
\end{itemize}
It was advised to look into \enquote{Doxygen} tool's auto-generated documentation of the software that captures the code and function level details of the Mediawiki software architecture.
As seen in \autoref{fig:doxygen}, it is clear that the the architecture details are captured at \enquote{Module}, \enquote{Class} and \enquote{File} level.
\vspace{10mm}
\begin{figure}[H]
  \centering
  \includegraphics[width=\textwidth]{images/doxygen}
  \caption[Auto-generated doxygen documentation]{Auto-generated doxygen documentation.}\label{fig:doxygen}
\end{figure}
Having realized the need for improved documentation of Mediawiki software architecture, the conscious demand arose for an improved process that could ensure the creation and maintenance of improved documents. Hence the scope of this thesis work was limited and set to the defined scope of \enquote{Improving the Software Architecture Documentation process of Mediawiki Software}.
\item \textbf{Organizational structure : } When understanding a software system as an organization and the processes that drive its daily activities, it is important to understand the \enquote{Who, What and How ?} of the system. This helps to grasp the organizational behavior as a complex socio-technical system. Thus the initial phase of the thesis involved the identification and understanding of current roles, responsibilities and processes that are practiced in the Mediawiki community. The outline of the current state-of-the-art organizational components have been captured in \autoref{chapter:ResearchQuestions} Section 2.3. 
\item \textbf{Documentation process as a part of the software process : } Every software development organization follows a process in order to manage, co-ordinate and streamline its daily SDLC (Software Development Life-Cycle) activities. Documentation itself is a part of this development process. Thus, it was important to analyze the process for generating and maintaining documentation by mediawiki community in order to assess its shortfalls and required improvements.
\indent The software and documentation process components and an outline of their basic interaction can be seen in the figure \autoref{fig:MWProcess}  The important roles within the community that are a part of the software process have been captured in the blue ellipses. The interaction of these roles with the system components and the activities involved as a part of individual responsibility has been captured in the diagram. 
\vspace{10mm}
\begin{figure}[H]
  \centering
  \includegraphics[width=\textwidth]{images/MWProcess}
  \caption[Mediawiki Software Process including Documentation process]{Mediawiki Software Process including Documentation process.}\label{fig:MWProcess}
\end{figure}

\item \textbf{Understanding wiki (Mediawiki.org) : }It was suggested that in order to understand the wiki platform provided by \enquote{mediawiki.org}, it was important to use various features like templates, extensions, visualizations, etc. that are used for rendering and structuring the content with better readability and navigability. Also, it is useful to understand the \enquote{namespaces} like \enquote{categories} that can help to organize the wiki pages into more understandable linearly-hierarchical structure. \enquote{Discussion pages} helps to improve the documentation where suggestions are given.
\\\indent Other possibilities of documentation are also available on Mediawiki that prove the fact that conscious efforts have been made towards creation and maintenance of documentation in general and software architecture documentation in particular. 

\begin{itemize}

\item Suggestions from 3\textsuperscript{rd} party Mediawiki discussion pages can be used to create new requests that can be linked to \enquote{phabricator tasks} \cite{3P_users}
\item \enquote{Project:PD} intends to create documentation as help pages that reach out to the public domain \cite{Project_pd}.
\indent The idea of this project is to provide a set of pages which can be copied into a fresh wiki installation, or included in the Mediawiki distribution. This will include basic user information and other \enquote{Meta information}, in a reasonably concise form. The basic concept is to create a compressed user guide which should focus on what users want and not explain other functions.
\item Mediawiki manual on coding conventions mandates having a \enquote{.txt} file in \enquote{\textbackslash{docs}} folder \cite{manual_cc}

\end{itemize}
\end{enumerate}

During these preparatory activities, different versions of the solutions were conceptualized and reasoned for their applicability and usability within the existing socio-technical environment of Mediawiki.
Arguments made for several concepts, judging the user scenarios and assessing the scope and feasibility of the concept helped in the decision-making process for the final solution.


\subsection{Identifying Use case scenarios }
In order to understand the documentation process and intended improvements, the use cases for document creation and maintenance activities were developed.
\newline \newline
\begin{mdframed}[leftmargin=10pt,rightmargin=10pt]
\textbf{Use Case 1 : Task for documentation is created by developer as a sub-task of code development}
\newline\newline \indent \textbf{User} : Developer / Reviewer
\newline \newline \indent \textbf{Activity} : task to write code and add document on \enquote{mediawiki.org}
\newline \newline \indent \textbf{System} : Phabricator
\newline \newline \indent \textbf{Task Details} : Documentation task is added by code developer / reviewer. The task is tagged with the related code patch git commit id.
\end{mdframed}

\begin{mdframed}[leftmargin=10pt,rightmargin=10pt]
\textbf{Use Case 2 : Developer writes and commits a piece of code}
\newline \newline \indent \textbf{User} : Developer
\newline \newline \indent \textbf{Activity} : Software development - write code
\newline \newline \indent \textbf{System} : Mediawiki
\newline \newline \indent \textbf{Task Details} : A usual development activity to add code which is then review and pushed into production. Documentation may or may not be created or updated
\end{mdframed}

\begin{mdframed}[leftmargin=10pt,rightmargin=10pt]
\textbf{Use Case 3 : Only document is added}
 \newline \newline \indent \textbf{User} : Documenter
 \newline \newline \indent \textbf{Activity} : Creating / updating / reviewing software architecture documentation of mediawiki.
 \newline \newline \indent \textbf{System} : Mediawiki
 \newline \newline \indent \textbf{Task Details} : A usual documentation activity to add software architecture documents for which may or may not be a part of maintenance activities
\end{mdframed}

\begin{mdframed}[leftmargin=10pt,rightmargin=10pt]
\textbf{Use Case 4 : Documentation task on phabricator}
\newline \newline \indent \textbf{User} : Developer / Reviewer
\newline \newline \indent \textbf{Activity} : Creating / updating / reviewing software architecture documentation of mediawiki.
\newline \newline \indent \textbf{System} : Phabricator
\newline \newline \indent \textbf{Task Details} : Task to add software architecture documents is created under the project \enquote{MediaWiki-Documentation} \cite{MW_doc_phab} and tagged as \enquote{documentation} to link all the tasks under this project. This task may or may not be assigned . Open comments section serves as a discussion forum to find related tasks or find people to complete the task.
\end{mdframed}
These use cases clearly identify the responsibilities as a part of current process and need to be considered for their roles in the improved process as well. The improvement will not require the complete change in roles and responsibilities. Rather, the same use cases need to be satisfied with a better process.
\newline
\\\indent In the \autoref{fig:use-case-scenario} all the above use case have been considered in a single use-case diagram to visually understand the actors and their activities in the system. The task management system refers to the \enquote{Phabricator} as a system with which the various actors interact.

\begin{figure}[H]
  \centering
  \includegraphics[width=\textwidth]{images/use-case-scenario}
  \caption[Use-case scenarios explaining user roles and tasks]{Use-case scenarios explaining user roles and tasks.}\label{fig:use-case-scenario}
\end{figure}

In the Table \autoref{Role-doc-maintenance} the previously identified \enquote{roles} in the documentation process have been listed as per the use-case scenarios. The column \enquote{Maintainability} captures the possibility and scope of documentation maintenance. As understood by the responsibility field, the use cases where maintenance is \enquote{possible}, human involvement is obligatorily required as a part of the task cretion and documentation process. This means that the Developer / Reviewer needs to be actively involved in the regular review and creation of tasks whenever the software architecture (source code) is being developed.

\begin{table}[]
\centering
\caption{Maintenance of documentation in different user scenarios}
\label{Role-doc-maintenance}
\begin{tabular}{@{}
>{\columncolor[HTML]{F8A102}}c |
>{\columncolor[HTML]{FFFFC7}}c |
>{\columncolor[HTML]{FFFFFF}}l |
>{\columncolor[HTML]{FFFFC7}}c |@{}}
\toprule
\multicolumn{1}{l|}{\cellcolor[HTML]{F8A102}{\bf \parbox{2cm}{Use Cases}}} & \cellcolor[HTML]{FFFC9E}{\bf \parbox{2cm}{Role}} & \multicolumn{1}{c|}{\cellcolor[HTML]{FFFC9E}{\bf Responsibility}} & \cellcolor[HTML]{FFFC9E}{\bf Maintainability} \\ \midrule
\multicolumn{1}{|c|}{\cellcolor[HTML]{F8A102}{\bf 1}}        & Developer/ Reviwer                 & \parbox{4cm}{Documentation task as a follow-on of development task}                                & possible                                      \\ \midrule
\multicolumn{1}{|c|}{\cellcolor[HTML]{F8A102}{\bf 2}}        & Developer                          & \parbox{4cm}{Coding (software development): branch / push / merge / commit}                                     & not possible                                  \\ \midrule
\multicolumn{1}{|c|}{\cellcolor[HTML]{F8A102}{\bf 3}}        & Documentor                         & \parbox{4cm}{Create documentation on wiki platform - mediawiki.org}                                 & not possible                                  \\ \midrule
\multicolumn{1}{|c|}{\cellcolor[HTML]{F8A102}{\bf 4}}        & Developer/ Reviewer                & \parbox{4cm}{Task specially created/ assigned for documentation}                                            & possible                                      \\ \bottomrule
\end{tabular}
\end{table}

\textbf{Requirement for the Improved Documentation process : }The above tabular categorization helps to understand the need for a semi-automated process where the developer is not completely burdened with the responsibility of review and maintenance of documentation on \enquote{mediawiki.org}.


\subsection{Assessing the Initial ideas }

This subsection helps to understand the initial ideas and the need to build upon them by discussing their pros-cons and feasibility of implementation within the thesis scope.


\begin{enumerate}

\item \textbf{Creating software architecture documents on \enquote{Mediawiki.org}}
\indent As a part of the literature survey Section 3.2.1, it was observed that documentation of wiki had several advantages over traditional documentation. But the concept of improved documentation process using only wiki as the platform may pose certain downsides. The following table compares the wiki with a version controlled platform on certain important criteria.

\begin{table}[]
\centering
\caption{Comparing wiki-documents and Version-controlled documentation }
\label{wiki-VC_compare}
\begin{tabular}{@{}
>{\columncolor[HTML]{FFFFFF}}l |
>{\columncolor[HTML]{FFFFC7}}l |
>{\columncolor[HTML]{FFFFFF}}l |@{}}
\toprule
{\bf Requirement}                                                                     & \multicolumn{1}{c|}{{\bf \parbox{5cm}{Documentation on Wiki (Disadvantages)}}}                             & \multicolumn{1}{c|}{{\bf \parbox{5cm}{Version Controlled documentation (Advantages)}}}                                   \\ \midrule
\multicolumn{1}{|c|}{\cellcolor[HTML]{F8A102}}                                        & \parbox{5cm}{Formal review is not possible. Anyone who has access to the wiki pages can edit and save pages.}                      & \parbox{5cm}{If source code is part of a review system then documentation also becomes part of the commits and is reviewed before final "push".} \\ \cmidrule(l){2-3} 
\multicolumn{1}{|c|}{\multirow{-2}{*}{{\bf Maintainability}}} & \parbox{5cm}{Tracking major changes is not possible.}                                                                              & \parbox{5cm}{The version control system provides novel solutions to identify textual differences.}                                               \\ \midrule
\multicolumn{1}{|c|}{\cellcolor[HTML]{F8A102}}                                        & \parbox{5cm}{Offline work is not possible.}                                                                                        & \parbox{5cm}{Coding / documentation can be performed online until the "commit" stage. Only the final "push" needs connectivity.}                 \\ \cmidrule(l){2-3} 
\multicolumn{1}{|c|}{\multirow{-2}{*}{{\bf Usability}}}       & \parbox{5cm}{Every page save creates a new history entry. An insignificant change may lead to unnecessary revision history entry.} & \parbox{5cm}{No history entry needs to be managed.}                                                                                              \\ \bottomrule
\end{tabular}
\end{table}

 The table \autoref{wiki-VC_compare} projects the cons of wiki and the pros of version control, hence, suggesting the need for version controlled documentation.
 
\item \textbf{Creating new \enquote{namespace} on \enquote{mediawiki.org} for Software Architecture Documentation}

\indent \enquote{Mediawiki.org} already provides many \enquote{namespaces} like \enquote{Manual} that are used for documentation as already mentioned in the previous chapters. Adding another namespace to this pool would add to the confusion of categorization and document organization This thesis aims at structured software architecture documentation as a part of the wiki page and does not aim to introduce unnecessary inclusions to \enquote{mediawiki.org}.
\\\indent The same categorization efficiency can be achieved by using the Mediawiki feature : \enquote{Category} instead of introducing a new namespace. The table \autoref{Category-namespace} captures a few advantages of categories in this regard.


\begin{table}[]
\centering
\caption{Comparing "Categories" and "Namespaces" for documentation pages categorization}
\label{Category-namespace}
\begin{tabular}{@{}
>{\columncolor[HTML]{F8A102}}c |
>{\columncolor[HTML]{FFFFFF}}l |
>{\columncolor[HTML]{FFFFFF}}l |@{}}
\toprule
\cellcolor[HTML]{FFFFFF}{\bf }                                  & \multicolumn{1}{c|}{\cellcolor[HTML]{FFFC9E}{\bf Namespace (Disadvantages)}}                                                                                                                                                & \multicolumn{1}{c|}{\cellcolor[HTML]{FFFC9E}{\bf Category (Advantages)}}                                                                              \\ \midrule
\multicolumn{1}{|c|}{\cellcolor[HTML]{F8A102}{\bf Creatioin}}   & \multicolumn{1}{|c|}{\parbox{5cm}{Namespace cannot be directly added as a special page (feature not available yet). It needs to be added along with the namespace array index to "LocalSeetings.php" file in a Mediawiki installation}                       }  & \multicolumn{1}{|c|}{\parbox{5cm}{Category can be easily added to the pool of categories via the wiki web page. It is equivalent to creating a new page on the wiki}}                    \\ \midrule
\multicolumn{1}{|c|}{\cellcolor[HTML]{F8A102}{\bf Description}} & \multicolumn{1}{|c|}{\parbox{5cm}{No explanation is available on the use and purpose of a particular namespace. Hence it may me confulsing and may lead to unintended use or categorization.}}                                                                  & \multicolumn{1}{|c|}{\parbox{5cm}{Category pages are like a usual page which can contain description of its purpose and usage and the list of other pages that belong to that category.}} \\ \midrule
\multicolumn{1}{|c|}{\cellcolor[HTML]{F8A102}{\bf Usage}}       & \multicolumn{1}{|c|}{\parbox{5cm}{It is difficult to handle pages under a namespace. The page has to be created with the right format (namespace:pageName). To change the namespace the existing page needs to be deleted an d a new page needs to be created}} & \multicolumn{1}{|c|}{\parbox{5cm}{It is easy to add, delete, update the category of a page. Only an edit page is required.}}                                                            \\ \bottomrule
\end{tabular}
\end{table}

This clarifies and explains the need to create a new category like \enquote{Software Architecture Documentation} for categorizing the intended documentation pages on \enquote{mediawiki.org}

\item \textbf{Provide guidelines for task management on Phabricator}
\indent An initial idea considered the provision of guidelines for stakeholders to \enquote{tag} documentation tasks on Phabricator and categorize them as \enquote{Software architecture documentation} task . These guidelines were meant to provide suggestions and helpful tips to identify potential documentation tasks as candidates for \enquote{software architecture document}. This would assist in creation of more meaningful tasks and result in production of more structured and complete architecture documents.
\\\indent But the idea of \enquote{guidelines for tagging software architecture documentation tasks} was not pursued for the following reasons :

\begin{itemize}
\item Difficult to define the guidelines within this thesis scope
\item Might not be complete (capture all architectural components, interfaces or modules )
\item Might be ambiguous in its purpose and use.
\item Cannot ensure that guidelines are followed.
\item External users may not be aware.
\item Guidelines can only suggest and not mandate a process
\item Location / placement of guidelines may be inaccessible/ unknown to all users. 
\end{itemize}
Hence, the idea of providing developers/ architects with a guideline for is not practical or feasible solution. Instead a more strict review process is required which ensures structure software architecture document creation and maintenance.


\item \textbf{Defining Responsibilities for the Role : Maintainer}

\begin{figure}[H]
  \centering
  \includegraphics[width=60mm]{images/role_Maintainers}
  \caption[The sphere of Maintainer's roles and responsibilities]{The sphere of Maintainer's roles and responsibilities.}\label{fig:role_Maintainers}
\end{figure}

An initial ides of document process improvement focused on the human \enquote{Maintainer} role to ensure a project-level co-ordination for documentation process. In \autoref{fig:role_Maintainers} it can be seen that the sphere of a Maintainer's responsibility encompasses the responsibilities of all the roles in the Software architecture documentation process. Thus, a dedicated \enquote{maintainer} can be a person who is an experienced developer or architect associated with particular architectural components with the task of documenting or reviewing the corresponding documents. A Maintainer's job is to periodically examine the software architecture document quality and availability on \enquote{mediawiki.org}. The table \autoref{human-bot} compares the human maintainer role to that of a Mediawiki BOT user role for responsibility assignment and handling.


\begin{table}[]
\centering
\caption{Comparing "Human-maintainer" role and "BOTs" for documentation maintenance responsibility}
\label{human-bot}
\begin{tabular}{@{}
>{\columncolor[HTML]{F8A102}}c |
>{\columncolor[HTML]{FFFFFF}}l |
>{\columncolor[HTML]{FFFFFF}}l |@{}}
\toprule
\cellcolor[HTML]{FFFFFF}{\bf }                                  & \multicolumn{1}{c|}{\cellcolor[HTML]{FFFC9E}{\bf Human-Maintainer (Disadvantages)}}                                                                                                                                                & \multicolumn{1}{c|}{\cellcolor[HTML]{FFFC9E}{\bf Mediawiki BOT (Advantages)}}                                                                              \\ \midrule
\multicolumn{1}{|c|}{\cellcolor[HTML]{F8A102}{\bf Role}}   & \multicolumn{1}{|c|}{\parbox{5cm}{Maintainer is not a dedicated user role in the wiki community. }}  & \multicolumn{1}{|c|}{\parbox{5cm}{A dedicated BOT user can be created as a Maintainer}}                    \\ \midrule
\multicolumn{1}{|c|}{\cellcolor[HTML]{F8A102}{\bf Responsibility}} & \multicolumn{1}{|c|}{\parbox{5cm}{Since a human maintainer is not a dedicated role, the person is vested with multiple responsibilities apart from documentation maintenance }}                                                                  & \multicolumn{1}{|c|}{\parbox{5cm}{A Documentation maintenance BOT is a dedicated user with a single defined responsibility}} \\ \midrule
\multicolumn{1}{|c|}{\cellcolor[HTML]{F8A102}{\bf Process}}       & \multicolumn{1}{|c|}{\parbox{5cm}{Fitting a new human role in the existing community structure as a part of an improved process may be challenging}} & \multicolumn{1}{|c|}{\parbox{5cm}{On the other hand, the community is open in adopting BOTs for automated activities within the software process.}}                                                           
\\ \midrule
\multicolumn{1}{|c|}{\cellcolor[HTML]{F8A102}{\bf Activity}} & \multicolumn{1}{|c|}{\parbox{5cm}{The periodic maintenance task may not fit into the schedule of a user who executes multiple activities.}}                                                                 & \multicolumn{1}{|c|}{\parbox{5cm}{The BOT can be configured to run a tedious activity at a desired schedule and duration (repeatedly).}}
\\ \midrule
\multicolumn{1}{|c|}{\cellcolor[HTML]{F8A102}{\bf Efficiency}} & \multicolumn{1}{|c|}{\parbox{5cm}{Rigid - heavily dependent on project schedule, activities and other factors}}                                                                 & \multicolumn{1}{|c|}{\parbox{5cm}{Flexible - independent, configurable, reliable}}
 \\ \bottomrule
\end{tabular}
\end{table}

The idea of an architectural module / component owner as its \enquote{documenter} and \enquote{maintainer} is difficult to achieve and thus, a Bot provides a more practical solution.

\item \textbf{Building an extension for document maintenance}
\indent As initial suggestion from the Mediawiki developers, it was understood that the best way to understand and document the software architecture of Mediawiki was to build an \enquote{extension}. It could not only help to understand the interfaces of architectural components but also some of their intrinsic functionalities and complexities. This generated an idea for building a documentation maintenance extension. 

\begin{table}[]
\centering
\caption{Comparing "Mediawiki extensions" and "BOTs" for documentation maintenance activity}
\label{extension-bot}
\begin{tabular}{@{}
>{\columncolor[HTML]{F8A102}}c |
>{\columncolor[HTML]{FFFFFF}}l |
>{\columncolor[HTML]{FFFFFF}}l |@{}}
\toprule
\cellcolor[HTML]{FFFFFF}{\bf } & \multicolumn{1}{c|}{\cellcolor[HTML]{FFFC9E}{\bf Extensions (Disadvantages)}}                                                                                                                                                & \multicolumn{1}{c|}{\cellcolor[HTML]{FFFC9E}{\bf Mediawiki BOT (Advantages)}}                                                                              \\ \midrule
\multicolumn{1}{|c|}{\cellcolor[HTML]{F8A102}{\bf Setup}}   & \multicolumn{1}{|c|}{\parbox{5cm}{Complex setup. Requires database configuration, setting up localization, preparing autoloadable classes and defining additional hooks.}}  & \multicolumn{1}{|c|}{\parbox{5cm}{No setup is required within the Mediawiki engine}}                    \\ \midrule
\multicolumn{1}{|c|}{\cellcolor[HTML]{F8A102}{\bf Implementation}} & \multicolumn{1}{|c|}{\parbox{5cm}{Rigid : extensions should be implemented as subclasses of a MediaWiki-provided base class }}                                                                  & \multicolumn{1}{|c|}{\parbox{5cm}{Flexible :	 Bots have no such restrictions and inter-dependency with the Mediawiki engine}} \\ \midrule
\multicolumn{1}{|c|}{\cellcolor[HTML]{F8A102}{\bf Assistance}} & \multicolumn{1}{|c|}{\parbox{5cm}{NO feature of manual assistance is available. The code can be modified, but once added, the extension behaves independently}}                                                                 & \multicolumn{1}{|c|}{\parbox{5cm}{BOTs can be configured to add manual assistance to reduce chances of mass errors}}
 \\ \bottomrule
\end{tabular}
\end{table}


But, the \autoref{extension-bot} highlights the complexity involved in building a Mediawiki extension as compared to a creating a BOT user assigned with a specific activity. Thus, it strengthens the concept of using a maintenance BOT for improving the Mediawiki software architecture documentation process.
\end{enumerate}

\section{Improved Documentation Process}
\indent After assessing all the ideas and concepts in the previous section, the final concept for this thesis contribution was conceived.

\textbf{The Final Concept : }
\newline
\indent The final concept for improving the software architecture documentation process of mediawiki software is to build  a \enquote{Documentation Monitor} that can track the maintenance activity of documents on \enquote{mediawiki.org} with the use of BOTs and with the organization of developer's responsibilities
\\\indent The concept aims to solve issues identified previously and to streamline the community people into a process.
\\\indent The improved documentation process is oriented with the salient features and intrinsic activities of a standard software processes. These dimensions of documentation process are captured in the \autoref{fig:doc_process_dimensions} and explained in the \autoref{PorcDimension}.
\\\indent The following sub-sections explain the idea and concept behind the improved documentation process which involves the interaction and co-ordination of human maintainers and a maintainer BOT in order to achieve the intended purpose of structure, up-to-date, useful software architecture documents.


\subsection{Roles and Responsibility definition and co-ordination}

\indent The crux of every software organization process constitutes the personnel belonging to the organization who are actively involved in the execution of the process. To successfully implement and execute the activities within the improved documentation process at Mediawiki, it is important to outline and define the roles an responsibility of the community stakeholders. 
\\\indent The concept involves a task-centered collaboration approach where the responsibilities of the defined roles are bound by explicit and implicit tasks. 
\begin{itemize}
\item \enquote{Explicit tasks} refer to the role-bound tasks created and assigned on the \enquote{Phabricator} or the well defined responsibilities for a particular user-role.
\item \enquote{Implicit tasks} refer to the activities that are an intrinsic part of community and organizational responsibility (e.g) community collaboration and acceptance.
\end{itemize}
Defining new responsibilities for existing roles may pose certain challenges pertaining to the implicit tasks . These challenges refer to the social aspects of defining new responsibilities within an existing community structure.The key principles to address these challenges are [\cite{Michel_2014}
\begin{enumerate}
\item the self-organization of the community through task decomposition
\item an on-line community support based on social design principles and best practices
\item an open science process to enable unanticipated contributions
\end{enumerate}

But the challenges are implicitly handled by the organization of the explicit tasks as suggested by the above principles. 
\newline
\indent \textbf{Principle 1 : }  Gracefully handle challenges regarding assignment of responsibility  - Task decomposition by role-responsibility organization
\begin{figure}[H]
  \centering
  \includegraphics[width=100mm]{images/Process_Roles}
  \caption[Defining distinct roles and responsibilities in a process]{Defining distinct roles and responsibilities in a process.}\label{fig:Process_Roles}
\end{figure}

As seen in the \autoref{fig:Process_Roles} the key roles identified within Mediawiki for the improved documentation process are :
\begin{itemize}
\item Architect - outlines/ defines/ upgrades/maintains the Mediawiki software architecture
\item Developer - writes/ updates / creates / maintains the Mediawiki software.
\end{itemize}
As a part of the software architecture documentation process, these existing roles have been vested with the added responsibility (explicit tasks) pertaining to software architecture documents:
\begin{itemize}
\item Developer - the prime responsibility of the software developers as a part of the documentation process is to create (write) the software architecture documents as a part of their regular development activity. This means that they are responsible for maintaining the up-to-date architecture information.
\item Architect - Documentation is medium for architects to communicate the description of the architectural components and their design decisions and implementation to the developers. Architects could expect the developers to produce a documentation aligned to the recommended guidelines and agreed specifications for understandability among readers of the documentation. The architect, owing to his responsibilities and experience, is capable of checking whether description of the component (view, style) suits the overall documentation and can be published. Hence, as a part of the documentations process, the architect is responsible for reviewing the documents written by the developers before they are finally \enquote{pushed} and also published on \enquote{mediawiki.org}

\end{itemize}

\textbf{Principle 2 : }On-line community support helps to overcome the challenges of missing knowledge base.  
\indent The Mediawiki community is a part of the larger open source community where it has succeeded to prove its presence over the years of its existence. Based on its success metrics, it can be very well assumed that the social design principles and best practices are inherently followed within Mediawiki community. Thus, introduction of an improved documentation process would be aided and assisted by the existing support structure.
\newline \newline
\textbf{Principle 3 : } Unanticipated contributions are a challenge in face of the socio-technical environment of open source software development:
\indent An open science (open knowledge base) process is the central paradigm of the Mediawiki community structure. Yet again, the improved documentation process will not be effected by this challenge. In fact, the external contributions in the form of open discussions will help improve the quality of the Mediawiki software architecture documents, which is the ultimate goal of the proposed concept

\subsection{Document Maintenance Bot - A proof of concept}
\indent The previous section capture the human resource organization aspect of the improved software architecture documentation process.
\\\indent Organizing roles-responsibilities helps to achieve better quality of software architecture documents. But, an important aspect of the improved documentation process is document maintainability on \enquote{mediawiki.org}. To streamline this maintenance process wherein, the documents on \enquote{mediawiki.org} are kept up-to-date, a more regular audit needs to be performed. Since this task is time consuming and adds to the burden of human maintainers, it is logical to use a BOT. \enquote{Bots are automated tools that can be used to perform tedious work or certain repetitive tasks related to a wiki} \cite{manual_bot}. This reasoning perfectly fits our requirement. Using a bot to take on the responsibility of maintainable / visible / sustainable documents, not only relieves the human effort but also regularizes the mandatory task of documentation.


\begin{figure}[H]
  \centering
  \includegraphics[width=\textwidth]{images/Bot_in_system}
  \caption[Introducing the doumentation maintenance BOT in the Mediawiki software process]{Introducing the doumentation maintenance BOT in the Mediawiki software process.}\label{fig:/Bot_in_system}
\end{figure}
The \autoref{fig:/Bot_in_system} introduces the BOT into the existing Mediawiki software and documentation process and shows the interactions between the different roles and systems that were captured in \autoref{fig:MWProcess}.
\indent The idea is to use the BOT as a maintenance assistant in the documentation process. As seen in the figure the BOT seamlessly interacts with different systems that are a part of the existing software process in order to improve the documentation process:
\begin{itemize}
\item Mediawiki software source code repository (/docs folder)
\item Pages on \enquote{Mediawiki.org}
\item Phabricator task management system
\end{itemize}
As shown by the dashed red lined, the BOT replaces a few activities and assists the human maintainer with his responsibilities.
This interaction and activities can be better-explained with a use-case scenario to understand the BOT's role in the documentation process. 
\indent Use Case : An open task is created when the pages on mediawiki.org is not updated as compared to the latest documents in the source code.
\\\indent  In \autoref{fig:MWDocBot} we can analyze this use-case as a sequence diagram where the BOT has been placed in the sequence of events in the process. The BOT is seen as the principal actor in this improved documentation process.
\begin{figure}[H]
  \centering
  \includegraphics[width=\textwidth]{images/MWDocBot}
  \caption[Maintenace Bot Sequence diagram]{Maintenace Bot Sequence diagram.}\label{fig:MWDocBot}
\end{figure}
Sequence Diagram in detail. 
\begin{enumerate}
\item \textbf{Pull} : The BOT initially pulls the latest version of the source code from the Mediawiki repository master branch. Once pulled, the text file containing the documentation within the source code is extracted
\item \textbf{Read} : The bot then reads the text from the corresponding documentation page on\enquote{Mediawiki.org}
\item \textbf{Compare} : The texts from source code and wiki page are compared for difference (plain text difference).
\item \textbf{Check History} :If there is a mismatch after comparison (web page is not up-to-date) then the bot checks the history of the wiki  page.
\item \textbf{Create Task} : If the wiki page was not updated in the last few days, a task is created in Phabricator for its maintenance (update/ create).
\end{enumerate}
This sequence of events initiated and executed by the BOT helps to maintain an updated copy of the documentation of mediawiki software architecture on \enquote{mediawiki.org}.  

\subsection{Guidelines for the process implementation and orientation}
When an improved process is being conceptualized, it is important to understand its usability and usefulness within the target system. In the case of Mediawiki software architecture documentation process, similar questions arise. It is important to understand how the process can be mandated or followed within the community.
\indent There are two scenarios where guidelines need to be provided in order to implement the improved process successfully \cite{Employee} :
\begin{itemize}
\item Orienting the existing community members to accept, understand and follow the process : 
\newline This scenario is a greater challenge with respect to the existing community and stakeholders. The improved documentation process needs to be mandated by organizational heads in order to include it within the existing software process. All current developers need to be advised and instructed by architects to obligatorily include the documentation of architectural components in the source code before committing the code that they write or update. 
\item Implementing the process by orienting new members to follow the process :
\newline Coding conventions and guidelines will help new developers and Mediawiki stakeholders to understand the importance of documentation and mandate them to include the architecture documents as a part of the source code.
\end{itemize}

\textbf{Suggested concept for ease in maintenance activity  :} Add area (architectural component) maintainer to the wiki page (e.g. in a template) which can be read by the Bot using the Mediawiki API in order to assign the Phabricator task for Software Architecture documentation to its responsible Maintainer\footnote{suggestion for better implementation from S Page (Mediawiki stakeholder) : https://en.wikipedia.org/wiki/User:S\_Page\_(WMF) }



\section{Dimensions of the Improved Documentation Process} \label{PorcDimension}
\indent This section highlights the key factors that contribute to the improvement of the documentation process. These factors have been identifies as dimensional facets of the documentation process structure. 
\begin{figure}[H]
  \centering
  \includegraphics[width=\textwidth]{images/doc_process_dimensions}
  \caption[Dimensions of documentation process features]{Dimensions of documentation process features.}\label{fig:doc_process_dimensions}
\end{figure}
\textbf{Capturing the dimensions of the improved documentation process in a nutshell} 

\begin{itemize}
\item \textbf{Review : } Any standard software process involves a strict review process at every stage of its lifecycle. Thus, it is important for the documentation process also to involve a review phase. This review involves the definition and interaction of the following:
\begin{enumerate}
\item Role - The role of a \enquote{Reviewer} has been defined as a part of the improved documentation process. Mediawiki clearly defines the user role of a reviewer for its software development process. The same or other users can be assigned the role of a \enquote{Document Reviewer}. An experienced developer or an architect is the best candidate for the role of a software architecture document reviewer.
\item Responsibility - The \enquote{Reviewer} has the responsibility of reviewing the software architecture documents when they are \enquote{pushed} as a part of the source code into the \enquote{Gerrit} review system. Only when the reviewer accepts/ approves the documentation, it is added to the \enquote{authoritative repository}. Once the document is approved, it can be copied/ added to \enquote{mediawiki.org}
\item Organization - The Roles and Responsibilities need to be assigned and defined as a part of the documentation process. This dimension of software project organization is ensured by the existing \enquote{Gerrit} Review system.
\end{enumerate}
\item \textbf{Maturity} - A higher maturity of a software process ensures the quality of a software and its organization
\begin{enumerate}
\item Compatibility - A standard consistent process that fits in with the existing matured processes is considered matured enough for its intended purpose. In case of the improved documentation process, it is compatible with the existing software process, with interfaces that blend into the existing system.
\item Operability - A user's effort in the operation control of a process determines its operarbility \cite{Berander2005}. In case of the improved documentation process, maintenance operation effort is shared by the BOT, thus making it highly operable.
\end{enumerate}
\item \textbf{Community} - The improved documentation process targets the open source community iin general and the Mediawiki software in particular. Pertinent to the community, it is important that the process is conceptualized bearing all the socio-technical factors in mind. 
\begin{enumerate}
\item Acceptance - A process serves its purpose when it is readily accepted in the community for which it has been defined and conceptualized. As the improved documentation process satisfies the stakeholder requirements, its acceptance is guaranteed within the community. 
\item Adaptability - To evaluate whether an improved process will be sustained within the community over a period of time, its adaptability needs to be assessed. Since the suggested process is an improvement over the existing process, it always leaves scope for further improvement as and when the software evolves.
\item Adoptability - Acquiring and fitting in a new/ improved process refers to the adoption of the process within the community. In the current scenario, the need and demand for software architecture documentation within Mediawiki suggests that the artifacts are highly desired and the process to maintain them will be adopted readily by the stakeholders. 
\end{enumerate}
\end{itemize}

\paragraph{This chapter concludes with a proof of concept based on the building, refining and evaluation of ideas. The summarization of the conceptualization phase can be as follows : The efficient task distribution based on role-responsibility definition and employment of human as well as non-human effort can result in an effective process improvement and a matured software standard. The following chapter implements the final concept devised in this chapter to realize a deployable solution for an improved documentation process for the Mediawiki software architecture. }