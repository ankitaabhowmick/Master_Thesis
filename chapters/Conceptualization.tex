\chapter{Conceptualization}\label{chapter:Conceptualization}

\section{Idea Generation and Evolution}
\indent This chapter outlines the approach that was taken to find answers to the research questions and conceptualize an optimal solution.

\subsection{Initial Ideas}
\indent Initial discussion were held with the developers / architects at Mediawiki to come to an understanding of their current community setup and to get a beginner's guide to the system. Following their suggestions and the advise of experienced developers, the following approaches were taken to kick-start the conceptualization phase.
 
Analyse software architecture documentation
Doxygen – tool for documenting S.A
Who, what how ?
Methodology for documentation
Understand feature of mediawiki.org - template, extension, visualizations
Different versions - arguments and decision-making, user scenarios discussions



Using wiki for documentation disadvantage:
Review not possible
offline work not possible
Tracking changes not possible
A page save creates a new history entry. This may lead to unnecessary revision history 



\section{Improved Process}
Documentation health monitor
Review of documents in a process-oriented structure - Gerrit review 

\subsection{Roles and Responsibility definition and co-ordination}

Key principles to address challenges of the task-centered collaboration approach are [Felix Master thesis]
1.) the self-organization of the community through task decomposition, 
2.) an on-line community support based on social design principles and best practices and
 3.) an open science process to enable unanticipated contributions

	Solves issues identified previously
 	
 	streamline more people into a process
 	
 	
\subsection{Document Maintenance Bot}
Using bot to take on the responsibility of 
maintainable/ visible


\subsection{A proof of concept}
figure : Bot in the system

In \autoref{fig:MWDocBot}  we can see the 
\begin{figure}[H]
  \centering
  \includegraphics[width=\textwidth]{images/MWDocBot}
  \caption[Maintenace Bot Sequence diagram]{Maintenace Bot Sequence diagram.}\label{fig:MWDocBot}
\end{figure}