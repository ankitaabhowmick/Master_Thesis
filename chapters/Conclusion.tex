\chapter{Conclusion}\label{chapter:Conclusion}
\section{Answer to Research Questions}
The ultimate goal of the thesis work is to answer the research questions that were initially formulated with the intention of finding relevant and satisfactory answers to them. In this concluding chapter the materialization of a concept helps to answer those research questions. The \autoref{RQ-solution} points to those answers.
\begin{table}[]
\centering
\caption{Providing answers to the Research Questions}
\label{RQ-solution}
\begin{tabular}{@{}
>{\columncolor[HTML]{F8A102}}c |
>{\columncolor[HTML]{FFFFFF}}l |
>{\columncolor[HTML]{FFFFFF}}l |@{}}
\toprule
\cellcolor[HTML]{FFFFFF}{\bf }                                  & \multicolumn{1}{c|}{\cellcolor[HTML]{FFFC9E}{\bf Research Questions}}                                                                                                                                                & \multicolumn{1}{c|}{\cellcolor[HTML]{FFFC9E}{\bf Solution}}                                                                              \\ \midrule
\multicolumn{1}{|c|}{\cellcolor[HTML]{F8A102}{\bf RQ1}}   & \multicolumn{1}{|c|}{\parbox{5cm}{How SAD process can be improved for Mediawiki S/W ?}                       }  & \multicolumn{1}{|c|}{\parbox{6cm}{The chapter 4 on Conceptualization and chapter 5 on Implementation elaborates the idea behind an improved SAD process for Mediawiki}}                    \\ \midrule
\multicolumn{1}{|c|}{\cellcolor[HTML]{F8A102}{\bf RQ2}} & \multicolumn{1}{|c|}{\parbox{5cm}{What state-of-the-art documentation processes are available in the industry that can meet OSS community requirements?}}                                                                  & \multicolumn{1}{|c|}{\parbox{6cm}{The literature survey in chapter 3 identifies the already established processes and helps to build on ideas for the concept derived in Chapter 4 }} \\ \midrule
\multicolumn{1}{|c|}{\cellcolor[HTML]{F8A102}{\bf RQ3}}       & \multicolumn{1}{|c|}{\parbox{5cm}{What are the metrics of evaluation of SAD and how can quality of SAD be assured ?}} & \multicolumn{1}{|c|}{\parbox{6cm}{Chapter 6 on evaluation captures the quality measurement details of the improved process}}                                                           
\\ \midrule
\multicolumn{1}{|c|}{\cellcolor[HTML]{F8A102}{\bf RQ4}}       & \multicolumn{1}{|c|}{\parbox{5cm}{What specific requirements of Mediawiki stakeholders should be met by the improved documentation process ?}} & \multicolumn{1}{|c|}{\parbox{6cm}{Chapter 2 on requirement analysis covers these requirements and the chapter 4 explains how to implement them.}}                                                            \\ \bottomrule
\end{tabular}
\end{table}
\section{Challenges}
As this research paper comes to an end it is important to list the implications of this work. In this regard, it is also important to throw light upon some inherent challenges that may not be overcome as a part of the solution. 
\begin{itemize}
\item Acceptance within community - It is hard to predict and even harder to determine the acceptability of a new process within a community that is more oriented towards product delivery rather than process orientation.
\item Socio-behaviorial aspects of OSS community - OSS community is not a standard organizational structure and is governed by a complex socio-technical environment that is non-cohesive and loosely-driven. Thus process improvement may not prove to be sustainable over a long period of time.
\item Technical challenges - A ten year old wiki engine is comparable to a legacy system that may already be too difficult to clean up. For the purpose of software architecture documentation, the compexity of code and lack of previous documentation will pose challenge during the creation of new documents.
\end{itemize}

\section{Benefits of implemented solution} 4.3, 5.3.3, 5.3.4

\section{Concluding Remanks} 	
