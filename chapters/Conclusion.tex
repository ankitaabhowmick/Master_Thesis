\chapter{Conclusion}\label{chapter:Conclusion}
\section{Answer to Research Questions}
The ultimate goal of the thesis work is to answer the research questions that were initially formulated with the intention of finding relevant and satisfactory answers to them. In this concluding chapter the materialization of a concept helps to answer those research questions. The \autoref{RQ-solution} points to those answers within the scope of this document.
\begin{table}[]
\centering
\caption{Providing answers to the Research Questions}
\label{RQ-solution}
\begin{tabular}{@{}
>{\columncolor[HTML]{F8A102}}c |
>{\columncolor[HTML]{FFFFFF}}l |
>{\columncolor[HTML]{FFFFFF}}l |@{}}
\toprule
\cellcolor[HTML]{FFFFFF}{\bf }                                  & \multicolumn{1}{c|}{\cellcolor[HTML]{FFFC9E}{\bf Research Questions}}                                                                                                                                                & \multicolumn{1}{c|}{\cellcolor[HTML]{FFFC9E}{\bf Solution}}                                                                              \\ \midrule
\multicolumn{1}{|c|}{\cellcolor[HTML]{F8A102}{\bf RQ1}}   & \multicolumn{1}{|c|}{\parbox{5cm}{How SAD process can be improved for Mediawiki S/W ?}                       }  & \multicolumn{1}{|c|}{\parbox{6cm}{The chapter 4 on Conceptualization and chapter 5 on Implementation elaborates the idea behind an improved SAD process for Mediawiki}}                    \\ \midrule
\multicolumn{1}{|c|}{\cellcolor[HTML]{F8A102}{\bf RQ2}} & \multicolumn{1}{|c|}{\parbox{5cm}{What state-of-the-art documentation processes are available in the industry that can meet OSS community requirements?}}                                                                  & \multicolumn{1}{|c|}{\parbox{6cm}{The literature survey in chapter 3 identifies the already established processes and helps to build on ideas for the concept derived in Chapter 4 }} \\ \midrule
\multicolumn{1}{|c|}{\cellcolor[HTML]{F8A102}{\bf RQ3}}       & \multicolumn{1}{|c|}{\parbox{5cm}{What are the metrics of evaluation of SAD and how can quality of SAD be assured ?}} & \multicolumn{1}{|c|}{\parbox{6cm}{Chapter 6 on evaluation captures the quality measurement details of the improved process}}                                                           
\\ \midrule
\multicolumn{1}{|c|}{\cellcolor[HTML]{F8A102}{\bf RQ4}}       & \multicolumn{1}{|c|}{\parbox{5cm}{What specific requirements of Mediawiki stakeholders should be met by the improved documentation process ?}} & \multicolumn{1}{|c|}{\parbox{6cm}{Chapter 2 on requirement analysis covers these requirements and the chapter 4 explains how to implement them.}}                                                            \\ \bottomrule
\end{tabular}
\end{table}
\section{Challenges}
As this research paper comes to an end it is important to list the implications of this work. In this regard, it is also important to throw light upon some inherent challenges that may not be overcome as a part of the solution. 
\begin{itemize}
\item \textbf{Acceptance within community }- It is hard to predict and even harder to determine the acceptability of a new process within a community that is more oriented towards product delivery rather than process orientation.
\item \textbf{Socio-behavioral aspects of OSS community} - Open Source Software community is not a standard organizational structure and is governed by a complex socio-technical environment that is non-cohesive and loosely-driven. Thus process improvement may not prove to be sustainable over a long period of time.
\item \textbf{Technical aspects} - A ten year old wiki engine is comparable to a legacy system that may already be too difficult to clean up. For the purpose of software architecture documentation, the complexity of code and lack of previous documentation will pose challenge during the creation of new documents.
\end{itemize}

\section{Benefits of implemented solution} 
\indent As listed and explained in \autoref{PorcDimension} of this thesis the conceptualized solution and its implementation have several highlighting features that establish its usefulness and effectiveness as an improved process. The concept proves its benefits in terms of the \enquote{Review} imperative, Process \enquote{Maturity} and \enquote{Community}orientation.
\newline \newline The \autoref{capabilities} lists the capabilities of the Bot in terms of its functionality and broad scope for documentation maintenance assistance. Apart from the basic implementation the Bot is capable of many more extended features that could be molded according to the preferences of the developers / architects / area maintainers. 	
\newline \newline \autoref{advantages} highlights some of the distinguishing advantages of using the Bot-maintainer concept which assists human maintainers to manage documentation tasks.

\section{Concluding Remarks} 	
\indent A sincere effort has been made in this thesis work to understand the Mediawiki software architecture documentation deficits and its maintenance needs in order to improve the process of documentation. This work's implication is not just limited to prove its requirement and use within the Mediawiki community but also reaches out far to find application for a maintainable documentation process in the Open Source Software community. With extensive study of existing processes within the community, the improved documentation process has been tailored to not only meet the requirements of the community stakeholders but also provide a solution that is coherent with the existing processes and systems. The implementation and its evaluation portrays that the goals have been achieved and further improvements to this basic solution can be realized and adopted in order to attain a novel solution for an improved software architecture documentation process. 
\\\indent Since a process-oriented solution cannot be empirically measured for perfectness without practicing it over a period of time, its applicability and correctness can only be qualitatively measured through assessment and review. This evaluation has been successfully measured based on past experience and knowledge of the community stakeholders. Thorough critical assessment has proved that the expectations of stakeholders have been met in terms of process requirements and its implications on the existing system and process have been successful. This goes a long way to validate the fact that process improvement (documentation process) is possible within the socio-technical limitations of Mediawiki community and within the broader scope of open source communities.
