\chapter{Introduction}\label{chapter:Introduction}

\section{Motivation}\label{sec:motiv}
\indent A good software architecture is the focal point of an evolving software \cite{Garlan2011}.  To make this software maintainable, extendable and sustainable, a robust software architecture and a defined documentation process for this architecture are required \cite{crouch_stephen_software_2013}.
\\\indent Documentation is a factor that determines the quality of a software. A good software architecture documentation helps to understand, evaluate and communicate the various architectural decisions from different stakeholder viewpoints \cite{BachmannDocumentingSoftware2010}. Also, as the software evolves and its complexity and dependencies increase, the corresponding architecture documentation needs to be updated as well\cite{yeates_stuart_OSS_2008}.
\\\indent Standardized software processes and tools for Application Lifecycle Management provide structural support to a software engineering project's life-cycle. The quality of a software process directly affects the quality of the software {\cite{Fuggeffa1988}.
\\\indent Summing up, a standard process for documentation improves the quality of the documents and ultimately, the quality of the software itself.

\section{About the Topic}\label{sec:about}
\subsection{Process Improvement in General}
Literature and early studies have defined the need and scope for continuous process improvement in a given industrial environment.
\newline\indent The \autoref{fig:process_improvement} shows the Plan-Do-Check-Act paradigm of software process improvement that lays the foundational basis for the need for improvement and the approach that should be followed in order to structure the improvement of any industrial process. This scope of structured process improvement is very well tailored for today's software industry and will be used as the underlying approach to structure this thesis work.

\begin{figure}[H]
  \centering
  \includegraphics[width=100mm]{images/process_improvement}
  \caption[The PDCA (Plan-Do-Check-Act) Paradigm /cite{Gorschek2006}]{The PDCA (Plan-Do-Check-Act) Paradigm \cite{Gorschek2006}.}\label{fig:process_improvement}
\end{figure}

\subsection{Getting to the thesis topic}
\indent Open source softwares have distinguished themselves as the trendsetters in the field of software engineering in this era and have demonstrated advantages which are beyond comparison. But there are a few downsides to this approach of software development \cite{Scacchi2007}. When a software depends on its online community which is only virtually connected, it suffers due to issues like  \enquote{persistent identity, newcomer confusion, etiquette standards, leadership roles, and group dynamics} \cite{Kim:2000:CBW:518514}. In the pretext of software process, open source software communities can be categorized as loosely co-ordinated and less process-oriented \cite{Zhao2003}. They believe in \enquote{Do-ocracy} where there is more focus of doing (building) the software from small to big, rather than following a process-oriented strict software life-cycle management process. This leads to the basic scope of this thesis : Improving the process in an open source environment
\\\indent In the recent past, Mediawiki software (WMF Foundation) has grown to become one of the largest open source communities in the world. This prompted the choice for the candidate software for the thesis: Improving the process for Mediawiki software
\\\indent As discussed above, software architecture documentation is as important in the software project as the software architecture itself. With some background study, it was found that lack of documentation is one of the major downsides of open source development model \cite{6923128} \cite{Zhao2003}. Hence this thesis topic aims to find a proof of concept  and a theoretical reasoning that may prove helpful for Open Source community in general and in particular : Improving the software architecture documentation process of Mediawiki software.

\section{Research scope}\label{sec:scope}
The scope of the thesis has been reduced to maintenance of structured (wiki-formatted) software architecture documentation of Mediawiki that is available as a part of the source code on \enquote{mediawiki.org}.
\\\indent Moreover, a process has been defined and demonstrated that can be used as a basis for a process that can aide in maintenance of documents over a period of time. Coupling the existing review process and task management system, this documentation process is well-bound to the practices in the Mediawiki community and aims to win greater acceptance of the defined process. \cite{6923128}

\section{Reader's guide}\label{sec:guide}
\indent The next chapter will list the questions to which this thesis aims to provide an answer. This will help us understand our initial assumptions, the existing problems and the expected solution.
\\\indent The following chapter will present literature analysis giving theoretical proofs to explain the important concepts for this research and the reasoning to support the thesis work (\autoref{chapter:LiteratureSurvey}).
\\\indent Then, \autoref{chapter:Conceptualization} will show the approach followed to find a proper solution by conducting discussions and meetings with the stakeholders. The system design is also covered in this chapter. 
\\\indent The consecutive chapter will present a detailed description of the system implementation, defining all of its features (\autoref{chapter:Implementation}).
\\\indent With regards to \autoref{chapter:Evaluation}, the thesis focuses on evaluating the proposed solution by comparing it with the standard processes in the industry and also by evaluating stakeholder satisfaction
\\\indent Lastly, \autoref{chapter:Conclusion} will conclude the concepts of this work, its future scope and the answers to the initially proposed research questions.


