\chapter{Evaluation}\label{chapter:Evaluation}
\indent Evaluation of the quality of an improved process within a community establishes its usefulness and justifies the efforts put into its implementation and fulfillment.
\\\indent The QualOSS standard \cite{Shahin2014} suggests that the quality of an improved process can be assessed on grounds of various process qualities related to its performance, efficiency and effectiveness, recorded over a period of time. Similarly, the evaluation of Software Process Improvement (SPI) \cite{Unterkalmsteiner2011} effect on high volume of literature study identified the \enquote{Pre-post evaluation} as the most common evaluation strategy.  It also suggests the use of process metrics, questionnaires and interviews as effective evaluation methods. The following sections list the use of these strategies to evaluate the improved software architecture documentation process.

\section{Meetings and Discussions}
\indent Communication is the key to understanding the views and reviews of users
Stakeholder viewpoint - 
\begin{enumerate}
\item Is the process adequate for its intended purpose (purpose of improving documentation process) ?
\item What features of this documentation process are attractive ? What features may pose challenges?
\item Will the process be effective / sustainable over a period of time?
\end{enumerate}

Experience viewpoint -
\begin{enumerate}
\item What can be the problems in enforcement of a strict process?
\item Is the strict documentation process adoptable in the current socio-technical environment of the Mediawiki community.
\item Is there a scope to define "document maintainer" activities and assign these responsibilities to existing Roles of Developer/ Architect. 
\item Does the implementation of this process require huge efforts in terms of required resources - personnel and time? 
\item Can that effort be estimated ?
\item Is the need of this process equatable to the estimated effort? What precedes - need or effort ?
\end{enumerate}


\section{Survey}

Evaluation based on user survey :


\section{Evaluation }
This section list the static analysis strategies that evaluate the implemented process for software architecture documentation.
\subsection{Review Questions}
\indent Architects and architecture analysts are concerned regarding the conformance of the architecture documentation to the set standards \cite{BachmannDocumentingSoftware2010}. For our specific purpose, this review is performed by Mediawiki architects for checking the conformance of documentation to its standards and requirements. The conformance can be checked by answering the following questions\cite{BachmannDocumentingSoftware2010}
\begin{enumerate}
\item Does the Software architecture documentation contain appropriate administrative and overview data?
\item Is the documentation required by the organization ?
\item Who are the stakeholders and are their requirements for documentation met by the software architecture documents.
\item Does the document achieve its purpose?
\item Does the document provide an introductory information and sufficient information to assist the understanding of the architectures
\end{enumerate}
\subsection{Community-related quality metrics} 
The following quality metrics from the QualOSS standard can be used to and evaluate the documentation process \cite{5314237}
\begin{enumerate}
\item Maintenance Capacity - Provide resource for maintenance, continuous support and improvement
\item Sustainability - Maintenance over an extended period of time
\item Process Maturity - Achieve goal
\end{enumerate}

\subsection{Measure the success of the implemented solution}
\begin{enumerate}
\item \textbf{Maintenance efforts (costs vs. capacity) \cite{Shahin2014}} 
Validity, Investment, Cost of quality and cost of process improvement \cite{Gorschek2006}
\begin{itemize}
\item \textbf{Personnel}
\item \textbf{Time}
\end{itemize}
\item \textbf{Is the process adequate for its intended purpose?}
\item \textbf{Process features}
 need to be evaluated in order to evaluate the applicability and usefulness \cite{Fuggeffa1988}
\begin{itemize}
\item \textbf{Architecture tracking}
\item \textbf{Multiple user support}
\item \textbf{Capture and reason}
\end{itemize}
\end{enumerate}

\section{Comparison with othe OSS}
