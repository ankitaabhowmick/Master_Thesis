\chapter{Implementation}\label{chapter:Implementation}

\section{Assumptions}
\indent the architectural components have been identified and a \enquote{.txt} file exists for each component
\\\indent these files are written in wiki formatted text
\\\indent all corresponding docs are already available as of date on  \enquote{Mediawiki.org}
\\\indent annotations in the source code \enquote{@see} help to navigate to the corresponding text file in the docs folder.
\\\indent The corresponding pages on Mediawiki.org have restricted access (e.g) Protected pages \emph{https://www.mediawiki.org/wiki/Help:Protected\_pages}. Or, the architecture description could be a part of the non-editable section such that they cannot be modified by other Mediawiki BOTs or users.
\section{Architecture and Technical outline}

\begin{figure}[H]
  \centering
  \includegraphics[width=\textwidth]{images/BOT_comp_diag}
  \caption[Component Diagram of the Maintenance BOT]{Component Diagram of the Maintenance BOT.}\label{fig:BOT_comp_diag}
\end{figure}


\begin{figure}[H]
  \centering
  \includegraphics[width=\textwidth]{images/BOT_state_diagram}
  \caption[State Diagram of the Maintenance BOT]{State Diagram of the Maintenance BOT.}\label{fig:BOT_comp_diag}
\end{figure}

\section{Details of the implemention}

Python

Git API 

PywikiBot

Mediawiki API

Mediawiki labs

Phabricator API


\section{Tests}
1. File exists
File does not exist
2. Task created
3. Task not created

\subsection{What All can this BOT do ?}
capture the difference and print it on a template on the page


\section{Future implementation and General Implications }

		

